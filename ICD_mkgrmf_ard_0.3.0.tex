% Time-stamp: <2001-12-28 13:11:45 dsd> 
% MIT Directory: ~dsd/CXC/specs/ICD/
% CfA Directory: /dev/null
% File: ICD_1DLIB.tex
% Author: D. Davis
%
% (this header is ~dph/libidl/time-stamp-template.el)
% to auto-update the stamp in emacs, put this in your .emacs file:
%      (add-hook 'write-file-hooks 'time-stamp)
%====================================================================

\documentclass[twoside]{article}
%\documentstyle[11pt]{article}
\input epsf.sty
\usepackage[outerbars,dvips]{changebar}
\usepackage{hyperref}
\usepackage{graphicx}
\usepackage{array}

%\usepackage[dvips]{graphics}
%\usepackage{epsf}
\textwidth=6.5in
\textheight=8.9in
\topmargin=-0.5in
\oddsidemargin=0in
\evensidemargin=0in

\newcommand{\putdraft}{\special{!userdict begin /bop-hook{gsave 200 30
translate 65 rotate /Times-Roman findfont 216 scalefont setfont 0 0
moveto 0.9 setgray (DRAFT) show grestore}def end}}
 

\newcommand{\Remark}[1]{\marginpar
   {\fbox{\parbox{1.7in}{\raggedright\scriptsize#1}}}}

\newcommand{\Putline}{
%  \advance\textwidth-26pt
  \rule{\the\textwidth}{1pt}
%  \advance\textwidth+26pt
  }
\def\nodata{ ~$\cdots$~ }
\newcommand{\Note}[1]{
\begin{changemargin}{0.25in}{0in}
\advance\textwidth-26pt
\parbox{\textwidth}{      % testing...
\hfill\Putline\newline
%\parbox{\textwidth}{\small\sf#1}\\
{\small\sf#1}\\            % testing...
\Putline\newline
}       % testing...
\advance\textwidth+26pt
\end{changemargin}
}

%%%%%%%%%%%%%%%%%%%%%%%%%%%%%%%%%%%%%%%%%%%%%%%%%%%%%%%%%
% \putstring{x}{y}{angle}{scale}{gray}{string}
% gray: 0=black, 1=white

\newcommand{\putstring}[6]{
\special{!userdict begin /bop-hook{gsave #1 #2 translate
#3 rotate /Times-Roman findfont #4 scalefont setfont
0 0 moveto #5 setgray (#6) show grestore}def end}
}
%%%%%%%%%%%%%%%%%%%%%%%%%%%%%%%%%%%%%%%%%%%%%%%%%%%%%%%%%

%%%%%%%%%%%%%%%%%%%%%%%%%%%%%%%%%%%%%%%%%%%%%%%%%%%%%%%%%
%    To change the margins of a document within the document,
%    modifying the parameters listed on page 163 will not work. They
%    can only be changed in the preamble of the document, i.e, before
%    the \begin{document} statement. To adjust the margins within a
%    document we define an environment which does it:
      \newenvironment{changemargin}[2]{\begin{list}{}{
         \setlength{\topsep}{0pt}\setlength{\leftmargin}{0pt}
         \setlength{\rightmargin}{0pt}
         \setlength{\listparindent}{\parindent}
         \setlength{\itemindent}{\parindent}
         \setlength{\parsep}{0pt plus 1pt}
         \addtolength{\leftmargin}{#1}\addtolength{\rightmargin}{#2}
         }\item }{\end{list}}
%    This environment takes two arguments, and will indent the left
%    and right margins by their values, respectively. Negative values
%    will cause the margins to be widened, so
%    \begin{changemargin}{-1cm}{-1cm} widens the left and right margins
%    by 1cm.
%%%%%%%%%%%%%%%%%%%%%%%%%%%%%%%%%%%%%%%%%%%%%%%%%%%%%%%%%

\newcommand{\Header}[2]{
\pagestyle{myheadings}                   %%%%%%%%%%
\markboth{\bf \qquad #1 \hfill #2 \qquad}%%%%%%%%%%
         {\bf \qquad #1 \hfill #2 \qquad}%%%%%%%%%%
}

%%%%%%%%%%%%%%%%%%%%%% END dph useful macros %%%%%%%%%%--------------


%%%
%%% Look for occurrences of five pound characters: #####, to locate places
%%% where updates are necessary
%%%

%%%
%%% revision info
%%%
\newcommand{\Revision}{\mbox{\em%
%%%
%%% ##### Update the revision information
%%%
%Revision 0.0---16 Mar 2001 % my first draft, uncirculated
%Revision 0.1.1---27 Mar 2001 % Format change incorporated
%Revision 0.1.2---02 May 2001 % Four Columns added
%Revision 0.2.1---28 Dec 2001 % Comments incorperated
Revision 0.3.0---15 Jan 2021 % Fix errors in functions, improve explanation on EE_FRACS
%Revision 1.2.0   15 Dec 1999 % Comments incorperated
%Revision 1.0---02 Feb 1998 % reviewed, updated.
}}



\hyphenation{pipe-line}
\hyphenation{pipe-lines}

%%%%%%%%%%%%%%%%%%%%%%%%%%%%%%%%%%%%%%%%%%%%%%%%%%%%%%%%%%%%%%%%%%%%%%%%

\Header{Data Product Interface Document: /}{\Revision}
\begin{document}


%\putstring{x}{y}{angle}{scale}{gray}{string}
% gray: 0=black, 1=white
%\putstring{70}{40}{90}{40}{0.90}{D R A F T -DRAFT- D R A F T}
%%% ##### comment the following line out for actual releases
%\putdraft

%%%%%%%%%%%%%%%%%%%%%%%%%%%%%%%%%%%%%%%%%%%%%%%%%%%%%%%%%%%%%%%%%%%%%%%%
%%%
%%% title stuff, no need to change anything
%%%

\begin{titlepage}

  \begin{tabular}{p{0.5\textwidth}>{\raggedleft}p{0.5\textwidth}}
    \includegraphics[height=8cm]{Kavli_Logo} & \includegraphics[height=8cm]{cxc-logo}
  \end{tabular}
  
  \begin{center}
    \vspace*{.5in}     
    {\Huge\bf Chandra X-ray Center}

    \vspace*{1in}

    {\LARGE\bf mkgrmf LSF Input Data:}

    \vspace*{.2in}
    {\LARGE\bf Data Product Interface Document:}
    \vspace*{.2in}
    
    \Revision
    
    \vfill

    \begin{tabular}{|l|l|}
      \hline
      version & author\\\hline\hline
      0.3.0 & H. M. G\"unther\\\hline
      0.2.1 & David S. Davis\\\hline
      0.1.2 & David S. Davis\\\hline
      0.1.1 & David S. Davis\\\hline
    \end{tabular}
  \end{center}
  \begin{tabular}{ll}
    \textbf{To:} & Jonathan McDowell, SDS Group Leader\\
    & Janet DePonte Evans, DS Software Development Manager\\
    \end{tabular}

\end{titlepage}


%%%%%%%%%%%%%%%%%%%%%%%%%%%%%%%%%%%%%%%%%%%%%%%%%%%%%%%%%%%%%%%%%%%%%%%%
%%%%%%%%%%%%%%%%%%%%%%%%%%%%%%%%%%%%%%%%%%%%%%%%%%%%%%%%%%%%%%%%%%%%%%%%
%%%%%%%%%%%%%%%%%%%%%%%%%%%%%%%%%%%%%%%%%%%%%%%%%%%%%%%%%%%%%%%%%%%%%%%%
%%%
%%% update info
%%%
\pagenumbering{roman}\setcounter{page}{2}

%%%
%%% ##### update as necessary
%%%

\begin{center}
\begin{tabular}{|c|c|c|p{2.5in}|} \hline
\multicolumn{4}{|c|}{}\\[1mm]
\multicolumn{4}{|c|}{\bf Document and Change Control Log}\\[3mm]\hline
{\bf Date} & {\bf Version} & {\bf Section} & {\bf Status} \\ \hline
19 Mar 01& 0.1.0& all&Initial Draft \\\hline
23 Apr 01& 0.1.1& all&Format Changes \\\hline
02 May 01& 0.1.2& all&Four columns added \\\hline
21 Sep 01& 0.2.0& all&Revised \\\hline
28 Dec 01& 0.2.1& all&Revised \\\hline
15 Jan 21& 0.3.0& all & Bring definitions of functions in line with CIAO practice, add ACIS-I and HRC-I as detectors. \\\hline
%04 Feb 99& 1.1.0& all&Revived as a separate document \\\hline
%30 Apr 99& 1.1.1& 4.1&Added enumerated axis or radial coord\\\hline
%30 Apr 99& 1.1.1& 4.1&changed TTYPE names\\\hline
%14 Jun 99& 1.2.0& all&Revised document using comments from DPH,KJG,AR 
%and JCM\\\hline
%
\hline
%
\end{tabular}
\end{center}

\section*{Unresolved Issues}

The following is a list of unresolved, un-reviewed, or un-implemented
items: 

\begin{enumerate}

\item Defocused data is not yet addressed

\end{enumerate}

\clearpage


%%%%%%%%%%%%%%%%%%%%%%%%%%%%%%%%%%%%%%%%%%%%%%%%%%%%%%%%%%%%%%%%%%%%%%%%
%%%%%%%%%%%%%%%%%%%%%%%%%%%%%%%%%%%%%%%%%%%%%%%%%%%%%%%%%%%%%%%%%%%%%%%%
%%%%%%%%%%%%%%%%%%%%%%%%%%%%%%%%%%%%%%%%%%%%%%%%%%%%%%%%%%%%%%%%%%%%%%%%
%%%
%%% table of contents, list of tables
%%%

\tableofcontents
\clearpage
% \listoftables
% \clearpage

%%%%%%%%%%%%%%%%%%%%%%%%%%%%%%%%%%%%%%%%%%%%%%%%%%%%%%%%%%%%%%%%%%%%%%%%
%%%%%%%%%%%%%%%%%%%%%%%%%%%%%%%%%%%%%%%%%%%%%%%%%%%%%%%%%%%%%%%%%%%%%%%%
%%%%%%%%%%%%%%%%%%%%%%%%%%%%%%%%%%%%%%%%%%%%%%%%%%%%%%%%%%%%%%%%%%%%%%%%

\pagenumbering{arabic}
\section{Introduction}

This document describes the interface to be employed in accessing 
the PSF Library products, according to the requirements stipulated in 
Applicable Document~\ref{appdoc:se03}.%
%

\subsection{Purpose}

The purpose of this document is to define the input data for the Line
Spread Function (generically referring to LSF) Library to be used in
Level 2 processing for the grating RMF generator (mkgrmf).
%

\subsection{Scope}

This interface shall apply to all Grating specific LSF
data products used by the tool {\tt mkgrmf}
and distributed to the CXC Data
Archive during the course of
the Chandra mission.

\section{Applicable Documents}
The Applicable Documents required for background and detail on
grating products are as follows:
\begin{enumerate}

\item\label{appdoc:data-prod} 
  AXAF Data Products Guide:\newline
  \url{https://cxc.cfa.harvard.edu/ciao/data_products_guide/}
\item\label{appdoc:coord}
  AXAF Coordinate Systems:\newline
  \url{https://cxc.cfa.harvard.edu/ciao/manuals.html} (see section General - Chandra Coordinate Systems)
\item\label{appdoc:se03}
 ASC AMO-2400 (SE03):\newline
 ASC Data System Requirements (ASC.302.93.0008)
\item\label{appdoc:ds01}
 ASC AMO-2401 (DS01) \newline
 ASC Data System Software Design (ASC.500.93.0006)
\item\label{appdoc:fitsstd}
  HEASARC FITS Standards:\newline
  \url{https://heasarc.gsfc.nasa.gov/docs/heasarc/ofwg/ofwg_recomm.html}
\item\label{appdoc:ascfits}
  ASC FITS File Designers' Guide:\newline
  \url{https://cxc.cfa.harvard.edu/contrib/arots/fits/ascfits.ps}
\item\label{appdoc:heasarccaldb}
  HEASARC FITS CALDB Standards:\newline
  \url{https://heasarc.gsfc.nasa.gov/docs/heasarc/caldb/caldb_doc.html}
\item\label{appdoc:asccaldb}
  AXAF CALDB Architecture\newline
  \url{https://cxc.cfa.harvard.edu/caldb/index.html}
\end{enumerate}

% synopsis for editing purposes...
%
%   1 appdoc:data-prod AXAF Data Products Guide
%   2 appdoc:coord     AXAF Coordinate Systems
%   3 appdoc:se03      ASC AMO-2400 (SE03)
%   4 appdoc:ds01      ASC AMO-2401 (DS01)
%   7 appdoc:ascfits   ASC FITS Designers' Guide
%   5 appdoc:fitsdef   NOST 100-1.1, Definition of the FITS
%
%   6 appdoc:fitsstd   HEASARC FITS Standards:
%   8 appdoc:hrcicd    HRC Data Products Guide:
%   9 appdoc:acisicd   ACIS Data Products Guide:


\section{Functional Description}

\subsection{Data Content Summary}

All Grating LSF Library files shall
consist of data files conforming to the FITS format (Applicable
Document~\ref{appdoc:fitsstd}). These files
contain header keyword entries and binary table (BINTABLE) extensions.
These files will contain a primary header, possibly null, followed
by a set of binary tables as described in Applicable
Document~\ref{appdoc:ascfits}. In addition, these files will conform to the 
HEASARC CALDB conventions (Applicable Document~\ref{appdoc:heasarccaldb}) and
have CONTENT, EXTNAME,and HDUCLASS keywords that conform to 
Applicable Document~\ref{appdoc:asccaldb}.


\subsection{Recipients and Utilization}

The primary recipients, via distribution from the archive, of the LSF
Library are Chandra observers, who will utilize these data products for
scientific data analysis. The CXC may also make use of specific LSF
Library data products for instrument calibration, instrument and/or
spacecraft monitoring and trends analysis, and validation and
verification of the Level 0, Level 1, and Level 1.5 software and of
the data products themselves.

\subsection{Pertinent Relationships with Other Interfaces}

Changes to the definition of CXC FITS, as described in Applicable
Documents~\ref{appdoc:ascfits}, may affect the format of the PSF data
products described in the current document.

\section{Assumptions and Constraints}

It is assumed that these products are placed into an exportable calibration database
(CALDB) for users. 

\subsection{Products Not Covered}

PSF products that are used for maintenance and diagnostic purposes
(those that are not supplied to the user for scientific data
analysis), or which are generic AXAF Level 2 products, are not
currently included within the interface defined by this document.

\subsection{Substructure Definition and Format}

The header components for the primary header and all binary table
extensions are defined and listed in the Applicable Documents. In 
general, the column or row numbers in the example FITS headers 
are arbirary unless otherwise indicated. It is the column name and
its attributes that specify the requirment. Additional columns not
specified here may be added to the file also long as they do not 
violate the interface. Software used to process the data can 
ignore the additional columns, copy them to the output file, or
optional use them for data processing. Likewise, HDU order is 
arbitrary, except for the primary HDU which must be first. HDUs
are intended to be referenced by name, not position. 

%%%%%%%%%%%%%%%%%%%%%%%%%%%%%%%%%%%%%%%%%%%%%%%%%%%%%%%%%%%%%%%%%%%%%%%%
\section{Access}
\subsection{Access Tools; Input/Output Protocol}

Since LSF Library products obey the formatting rules described in
Applicable Documents~\ref{appdoc:ascfits},
they may be accessed by any software that conforms to those standards,
including all versions of the FITSIO libraries that support the
BINTABLE extension. In addition, since they adhere to HEASARC
standards (Applicable Document~\ref{appdoc:fitsstd}), LSF data product
files are compatible with the input/output routines that constitute
the CXC data interface.

%\subsection{Timing and Sequencing Characteristics}

%%%%%%%%%%%%%%%%%%%%%%%%%%%%%%%%%%%%%%%%%%%%%%%%%%%%%%%%%%%%%%%%%%%%%%%%

\section{The File Structure for the Grating LSF Data}

This section describes the LSF parameter library which is a FITS file
containing the LSF fit parameters for the Chandra gratings. Each file
will contain the parameters for one grating type.
The * in the table below  denotes the ASC principal HDU.

\begin{table}[h]
{\footnotesize
\noindent\begin{tabular}{|llcccccp{1.2in}|}\hline
HDU
& \sc HDU Type
& \sc EXTNAME
& \sc EXTVER
& \sc CONTENT
& \sc HDUCLAS1
& \sc HDUCLAS3
& Description
\\
\hline
%
0 
& \sc NULL
& \nodata
& \nodata
& \nodata
& \nodata
& \nodata
& \nodata
\\
%
1 (*)
& \sc BINTABLE
& \sc MEG
& 1
& \sc LSF\_PARAM
& RESPONSE
& \sc LSF
& LSF coefficients
\\
%
\\\hline
%
\end{tabular}
}% close \small
\caption{File structure for the table with the LSF fit coefficients}
\label{tab:fstruct}
\end{table}
%

\subsection{File Names}

The filename convention shall be

$$\{ det\} \{grat\}\{order\}D \{date\} lsfparmN\{version\}.fits$$

where $\{det\}$ is one  $acis$ (for ACIS-S\footnote{Consistency would require this to be named $aciss$ but previous versions of this ICD used $acis$ because they did not consider ACIS-I, which is rarely used for grating observations. For backwards compatibility, the name $acis$ is continued to be used for ACIS-S.}), $hrcs$, $acisi$, or $hrci$ and $\{grat\}$ is one of $leg$, $meg$,
or $heg$. $\{order\}$ gives the diffraction order of the grating. Negative orders are prefixed with ``-''; positive orders have no sign, i.e. ``-1'' and ``1'' are the negative and positive first diffraction order, respectively.
$\{date\}$ follows the date convention in Applicable
Document~\ref{appdoc:ascfits}. 

As an example the lsf file for the HRC-S using the LETG grating should 
be of the form 

hrcsleg1D1999-07-22lsfparmN0000.fits

while the file for the MEG using ACIS-S should be 

acismeg1D1999-07-22lsfparmN0001.fits

\subsection{Column Descriptions}

\begin{table}[h]
\begin{center}
{\small
\begin{tabular}{|c|c|c|c|c|c|p{1.8in}|}
\hline
 & & & & & & \\
 \#
 & TTYPE
 & TUNIT
 & TFORM 
 & TLMIN
 & TLMAX
 & \multicolumn{1}{|c|}{Comment}\\
 & & & & & & \\\hline
%
 1
 & NUM\_WIDTHS
 & 
 & I
 & 0
 & \sc TBD
 & Number of Extraction Widths \\\hline
%
 2
 & WIDTH
 & Degrees
 & 3E
 & 0
 & \sc TBD
 & Extraction Width \\\hline
%
 3
 & NUM\_LAMBDAS
 & 
 & J
 & 0
 & \sc TBD
 & Number of wavelength points \\\hline
%
 4
 & TG\_LAM\_LO
 & Angstroms
 & nD
 & $0.0$
 & \sc TBD
 & Low wavelength of the extraction region\\\hline
%
 5
 & TG\_LAM\_HI
 & Angstroms
 & nD
 & $0.0$
 & \sc TBD
 & High wavelength of the extraction region\\\hline
%
 6
 & LAMBDAS
 & Angstroms
 & nD
 & $0.0$
 & \sc TBD
 & Input Photon Wavelength \\\hline
%
 7
 & EE\_FRACS
 & N/A
 & mD
 & $0.0$
 & $1.0$
 & Encircled Energy fraction\\\hline
%
 8
 & GAUSS$i$\_PARMS
 & N/A
 & kE
 & N/A
 & N/A
 & vector containing Gaussian parameters\\\hline
%
 9
 & LORENTZ$i$\_PARMS
 & N/A
 & kE
 & N/A
 & N/A
 & vector containing Lorentzian parameters\\\hline
%
 10
 & THETA\_MIN
 & degrees
 & E
 & 0
 & N/A
 & min off-axis angle\\\hline
%
 11
 & THETA\_MAX
 & degrees
 & E
 & 0
 & N/A
 & max off-axis angle for which this is valid\\\hline
%
 12
 & PHI\_MIN
 & degrees
 & E
 & 0
 & N/A
 & min azimuthal angle for which this is valid\\\hline
%
 13
 & PHI\_MAX
 & degrees
 & E
 & 0
 & N/A
 & max azimuthal angle for which this is valid\\\hline
%
%
\end{tabular}
}% close \small
\caption{Binary table with the LSF fit coefficients}
\label{tab:parms}
\end{center}
\end{table}%
The $i$ in parmeters 8 and 9 is currently set to 1 but multiple parameter 
sets are allowed. So for instance if future fits include a second
Gaussian component along with the current parameters the lsf file 
structure supports that as an additional column, e.g. GAUSS2\_PARM. 
Also n is the number of wavelengths, m is 3$\times$n, and k is 
3$\times$n$\times$NUM\_WIDTHS. 
%
\subsubsection{Comments on the Columns}

The {\tt NUM\_WIDTH} column give the number of widths for which the LSF  
was extracted and tabulated. In the current incarnation
this value is three for the MEG and HEG files. For the LETG only one 
width is currently implemented.  

The {\tt WIDTH} column gives the width, in degrees, of the extraction
region used to extract the LSF data. The next column gives the
number of wavelengths at which the LSF is tabluated. Currently these
must be the same for each extraction width and are assumed to be in
angstroms.  The columns {\tt TG\_LAM\_LO} and {\tt TG\_LAM\_HI} give
that lower and upper wavelength of the box that was used to extract
the LSF data. The {\tt LAMBDA} column gives the wavelength of the peak
position of the LSF.

The encircled energy fraction is tablulated in the {\tt EE\_FRAC}
column.  The vector column containing the EE\_FRAC is a
2-dimensional vector with the ee\_frac for each wavelength as a vector
and additional rows are for the ee\_frac for each different width as
defined above. This means that the dimensionality of the vector will
be num\_lambdas $\times$ num\_widths.


%$$ \sum_{j=0}^{num\_widths-1} \sum_{i=0}^{num\_lambdas-1}  
%{\rm ee\_vector}(i+j) = ee\_frac(i,j)$$
%
%Each of the vectors containing the LSF fits parameters are
%arranged as a matrix and are 3 x NUM\_LAMBDAS x NUM\_WIDTHS.
%For the Gaussian parameters
%
%$$ \sum_{i=0}^{num\_widths-1} \sum_{j=0}^{num\_lambdas-1} \sum_{k=0}^2
%{\rm Gaussian\_parms\_vector}(i+j+k) = {\rm gauss\_parms}(i,j,k)$$
%

The columns for the fit parameters shall be a matrix of
n$\times$j$\times$k where n is the number of parameters and 
their order is 

\begin{obeylines}
gauss\_parms(0,j,k) = Gaussian amplitude,

gauss\_parms(1,j,k) = Gaussian $\sigma$ in \AA,

gauss\_parms(2,j,k) = peak position in \AA.
\end{obeylines}

The number of elements is given by $j$ and the maximum value of $j$ is
NUM\_LAMBDAS. The number of widths is given by $k$. The function
parameters at a given j must be for  
the wavelength range TG\_LAM\_LO(j) and TG\_LAM\_HI(j) and the maximum
value of $j$ is given by NUM\_WIDTHS. 

%
%For the Lorentzian parameters
%
%$$ \sum_{i=0}^{num\_widths-1} \sum_{j=0}^{num\_lambdas-1} 
%{\rm \sum_{k=0}^2 Lorentzian\_parms\_vector}(i+j+k) = 
%{\rm lore\_parms}(i,j,k)$$
%

For the Lorenzian parameters the order of the parameters in the matrix
shall be

\begin{obeylines}
lorentzian\_parms(0,j,k) = Lorentzian amplitude,

lorentzian\_parms(1,j,k) = Lorentzian width in \AA,

lorentzian\_parms(2,j,k) = peak position in \AA,
\end{obeylines}
with the other elements obeying the convention for the Gaussian
parameters above. 

\subsection{Allowed Functional Forms}

The functional forms that are allowed for the LSF in mkgrmf are
currently a Gaussian and a Lorentzian. 

\begin{equation}\label{eqn:gauss}
  G(r) = \frac{A}{\sqrt{2\pi\sigma^2}} e^{\frac{-(r-r_0)^2}{2\sigma^2}}
\end{equation}

where $A$ is the amplitude, $\sigma$ is the Gaussian width in angstroms and $r_0$ is the
peak of this component of the LSF. 

\begin{equation}\label{eqn:lorentz}
  L(r) = \frac{A}{2\pi} \frac{FWHM}{(r - r_0)^2 + \left(\frac{FWHM}{2}\right)^2)}
\end{equation}

where $A$ is the amplitude, $r_0$ is the peak position of the Lorenztian component in questions, which need not
be the same any of the other compoments, and $FWHM$ is the
full width of the line profile at half the maximum. 

\subsection{Normalization of components}
All Gaussian and Lorentzian components have an ampitude defined in their parameters. However, the total normalization of each RMF is given by the \texttt{EE\_FRACS} column. That means that for LSFs defined by just a single component, the given component amplitude is arbitrary. For LSFs with multiple components the amplitude of each component only matters in a relative sense: After adding up all LSF components, the LSF is normalized such that the sum over all channels is the number given in the \texttt{EE\_FRACS} column. If, for example, all amplitudes $A$ are doubled but the \texttt{EE\_FRACS} values is constant, the resulting LSF does not change.


\subsection{Size Estimates}

The ASC primary extension of each file will have 13 columns 
to describe the LSF parameters at each energy. The vector columns
are real floating point numbers so the size of the file can be 
estimated by the number of entries $\times$ 11 (the number of 
vector columns). So for a typical MEG LSF file with $\sim$2500 entries
the size of the data area should be 27500 bytes (11$\times$2500). 

\section{FITS Header Templates}

The following header sections have been taken from the ASC 
FITS file specifications. The example FITS headers given 
here are examples only, the column numbers, axis numbers, and
keyword values are {\it not} necessarily those in the LSF data files. 

\subsection{Header Components with No Changes}

\begin{itemize}

\item{Mandatory component for the Image Primary Header (M)}

\item{Mandatory component for Null Primary Header (M)}

\item{Mandatory component for Binary Table extension (M)}

\item{Mandatory component for Image extension (M)}

\item{Full configuration control component (CC)}

\item{Short configuration control component (Short CC)}

\item{Configuration control component for null primary HDU (Null CC)}

\end{itemize}

\subsection{Header Components with Changes}

\subsubsection{Full Timing Component (T)}

\begin{verbatim}
DATE    = '0000-00-00T00:00:00' / Date and time of file creation
TIMEUNIT= 's       '           / Time unit
\end{verbatim}

\subsubsection{Short Timing Component (Short T)}

\begin{verbatim}
DATE    = '0000-00-00T00:00:00' / Date and time of file creation
TIMEUNIT= 's       '           / Time unit
\end{verbatim}

\subsubsection{Full Observation info Component}

\begin{verbatim}
MISSION = 'AXAF    '           / Mission
INSTRUME= 'ACIS    '           / Instrument
DETNAM  = 'ACIS-S  '           / Detector
GRATING = 'HETG    '           / Grating
GRATTYP = 'MEG     '           / Grating Type  
\end{verbatim}



\subsubsection{Grating Specific Components}
\begin{verbatim}
GRATING = 'HETG    '           /Grating in use
GRATTYP = 'HEG     '           /Grating data
TG_M    = 1
ORDER   = 1
CDTP0001= 'DATA    '           /Virtual data set
CCLS0001= 'CPF     '           /Basic Calibration file
CDES0001= 'MEG LSF parameters input for mkgrmf'/Description
CBD10001= 'GRATING(HETG)'
CBD20001= 'GRATTYPE(MEG)'
CBD30001= 'TG_M(1) '
CBD40001= 'SHELL(0011)'
CBD50001= 'ORDER(1)'
CVSD0001= '1999-07-22T00:00:00'
CVST0001= '00:00:00'
CCNM0001= 'LSFPARM'
CBD10001= 'Energy(5)keV'       /Energy range for PSF
CBD20001= 'THETA(0 - 2)arcmin'     /Distance from optical axis for PSF
CBD30001= 'PHI(0.0 - 360.)deg'        /Azimuthal angle for PSF

\end{verbatim}

%

%\section{Focal Plane positions of the LSF}


\end{document}
